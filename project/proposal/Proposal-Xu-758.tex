\documentclass[letterpaper, 10 pt, conference]{ieeeconf}  % 

\title{\LARGE \bf Project Proposal}

\author{Minqi Xu, 20845758, m259xu@uwaterloo.ca} % <-this % stops a space
% \thanks{$^{1}$ X and Y  are with the 
% School of Computer Science, University of Waterloo,
% 200 University Avenue, Waterloo, Ontario, Canada N2L 3G1.        {\tt\small }}%
% }


\begin{document}
\onecolumn
\maketitle
%
% \begin{abstract}
% \end{abstract}
%

\section{Topic}
% \label{sec:intro}
Generic Algorithm

% Citation examples: \cite{An82,An09,Go89,Ra10}.


% \section{Related Work}
% \label{sec:rw}

% \section{Algorithm Implementation}
% \label{sec:rw}

% \section{Results and Discussion}




\section{Introduction}

Genetic algorithm is a search and optimization algorithm inspired by 
the natural selection and genetics. It is commonly used to generate high
quality solution to optimization and search problems through some bionic
operations such as mutation, crossover, and selection. \cite{wikiga}. Here,
mutation introduces small random changes in the offspring's genetic information,
crossover involves combining genetic information from two individuals to create
offsprings.

In the algorithm, each generation consists of a population of individuals,
who contains a fitness score (which is used to demonstrate the ability to
"complete", or adapt the environment), and each individual represents a point
in search space and possible solution. In general, the algorithm is in the following
logic:
\begin{itemize}
    \item Individuals in a population compete for resources and mates
    \item Those who are "fittest" mate to produce more offspring than others
    \item "Genes" from "fittest" individual propagate throughout the generation, because of this, parents sometimes create offsprings that are better than themselves
    \item As a result, each generation is better suited to their environment. \cite{gfg}
\end{itemize}

\section{Connection to the course objectives}

While selecting potentially useful solutions for recombination, a sampling method called
Stochastic universal sampling (SUS) is introduced. It is an optimal sequential sampling 
algorithm with no bias, minimal spread, and achieve all N samples in a single tranversal. \cite{bj87} Compared to the Fitness Proportionate
Selection, SUS performs better when there exists an individual with an extremely large fitness
compared to others. \cite{wikisus} In Fitness Proportionate Selection (also known as Roulette Wheel
Selection), the fitness level for individuals (which is used to associate the probability of selection with each
 individual chromosome) is defined as the fitness of individuals divided by the total fitness of individuals
 in the generation. \cite{wikifps}

 \section{Application of Generic Algorithm}
 \begin{itemize}
    \item Recurrent Neural Network
    \item Mutation testing
    \item Code breaking
    \item Filtering and signal processing
    \item Learning fuzzy rule base \cite{gfg}
 \end{itemize}

 \section{Personal Reason for Choosing This Topic}
 The first time I came across the term Generic Algorithm was in the AI course last semester. My professor showed us
 an example of it in class. \cite{gaeg1}. I was attracted by the idea behind the algorithm immediately. A few weeks ago,
 in the lecture of security, the professor mentioned this algorithm again, and also showed us another example about Snake. \cite{gaeg2}
This algorithm has fascinated me since then. Because this algorithm has randomness, it just meets the requirement of this
project. So when I saw the requirement, my first thought was to do a project about Genetic Algorithm.


\bibliographystyle{IEEEbib}
\bibliography{template}

\end{document}



